% Options for packages loaded elsewhere
\PassOptionsToPackage{unicode}{hyperref}
\PassOptionsToPackage{hyphens}{url}
\PassOptionsToPackage{dvipsnames,svgnames,x11names}{xcolor}
%
\documentclass[
  12pt,
  letterpaper,
]{article}

\usepackage{amsmath,amssymb}
\usepackage{iftex}
\ifPDFTeX
  \usepackage[T1]{fontenc}
  \usepackage[utf8]{inputenc}
  \usepackage{textcomp} % provide euro and other symbols
\else % if luatex or xetex
  \usepackage{unicode-math}
  \defaultfontfeatures{Scale=MatchLowercase}
  \defaultfontfeatures[\rmfamily]{Ligatures=TeX,Scale=1}
\fi
\usepackage{lmodern}
\ifPDFTeX\else  
    % xetex/luatex font selection
\fi
% Use upquote if available, for straight quotes in verbatim environments
\IfFileExists{upquote.sty}{\usepackage{upquote}}{}
\IfFileExists{microtype.sty}{% use microtype if available
  \usepackage[]{microtype}
  \UseMicrotypeSet[protrusion]{basicmath} % disable protrusion for tt fonts
}{}
\makeatletter
\@ifundefined{KOMAClassName}{% if non-KOMA class
  \IfFileExists{parskip.sty}{%
    \usepackage{parskip}
  }{% else
    \setlength{\parindent}{0pt}
    \setlength{\parskip}{6pt plus 2pt minus 1pt}}
}{% if KOMA class
  \KOMAoptions{parskip=half}}
\makeatother
\usepackage{xcolor}
\usepackage[margin=1in]{geometry}
\setlength{\emergencystretch}{3em} % prevent overfull lines
\setcounter{secnumdepth}{5}
% Make \paragraph and \subparagraph free-standing
\makeatletter
\ifx\paragraph\undefined\else
  \let\oldparagraph\paragraph
  \renewcommand{\paragraph}{
    \@ifstar
      \xxxParagraphStar
      \xxxParagraphNoStar
  }
  \newcommand{\xxxParagraphStar}[1]{\oldparagraph*{#1}\mbox{}}
  \newcommand{\xxxParagraphNoStar}[1]{\oldparagraph{#1}\mbox{}}
\fi
\ifx\subparagraph\undefined\else
  \let\oldsubparagraph\subparagraph
  \renewcommand{\subparagraph}{
    \@ifstar
      \xxxSubParagraphStar
      \xxxSubParagraphNoStar
  }
  \newcommand{\xxxSubParagraphStar}[1]{\oldsubparagraph*{#1}\mbox{}}
  \newcommand{\xxxSubParagraphNoStar}[1]{\oldsubparagraph{#1}\mbox{}}
\fi
\makeatother


\providecommand{\tightlist}{%
  \setlength{\itemsep}{0pt}\setlength{\parskip}{0pt}}\usepackage{longtable,booktabs,array}
\usepackage{calc} % for calculating minipage widths
% Correct order of tables after \paragraph or \subparagraph
\usepackage{etoolbox}
\makeatletter
\patchcmd\longtable{\par}{\if@noskipsec\mbox{}\fi\par}{}{}
\makeatother
% Allow footnotes in longtable head/foot
\IfFileExists{footnotehyper.sty}{\usepackage{footnotehyper}}{\usepackage{footnote}}
\makesavenoteenv{longtable}
\usepackage{graphicx}
\makeatletter
\newsavebox\pandoc@box
\newcommand*\pandocbounded[1]{% scales image to fit in text height/width
  \sbox\pandoc@box{#1}%
  \Gscale@div\@tempa{\textheight}{\dimexpr\ht\pandoc@box+\dp\pandoc@box\relax}%
  \Gscale@div\@tempb{\linewidth}{\wd\pandoc@box}%
  \ifdim\@tempb\p@<\@tempa\p@\let\@tempa\@tempb\fi% select the smaller of both
  \ifdim\@tempa\p@<\p@\scalebox{\@tempa}{\usebox\pandoc@box}%
  \else\usebox{\pandoc@box}%
  \fi%
}
% Set default figure placement to htbp
\def\fps@figure{htbp}
\makeatother

\usepackage{fancyhdr}
\pagestyle{fancy}
\fancyhf{}
\rhead{NEU Seattle Devs}
\lhead{225 Building Environment Report}
\cfoot{\thepage}
\usepackage{xcolor}
\definecolor{darkblue}{RGB}{32, 38, 115}
\usepackage{sectsty}
\sectionfont{\color{darkblue}}
\usepackage{datetime}
\date{\today}
\usepackage[english]{babel}
\renewcommand{\today}{\monthname[\the\month] \the\day, \the\year}
\makeatletter
\@ifpackageloaded{caption}{}{\usepackage{caption}}
\AtBeginDocument{%
\ifdefined\contentsname
  \renewcommand*\contentsname{Table of contents}
\else
  \newcommand\contentsname{Table of contents}
\fi
\ifdefined\listfigurename
  \renewcommand*\listfigurename{List of Figures}
\else
  \newcommand\listfigurename{List of Figures}
\fi
\ifdefined\listtablename
  \renewcommand*\listtablename{List of Tables}
\else
  \newcommand\listtablename{List of Tables}
\fi
\ifdefined\figurename
  \renewcommand*\figurename{Figure}
\else
  \newcommand\figurename{Figure}
\fi
\ifdefined\tablename
  \renewcommand*\tablename{Table}
\else
  \newcommand\tablename{Table}
\fi
}
\@ifpackageloaded{float}{}{\usepackage{float}}
\floatstyle{ruled}
\@ifundefined{c@chapter}{\newfloat{codelisting}{h}{lop}}{\newfloat{codelisting}{h}{lop}[chapter]}
\floatname{codelisting}{Listing}
\newcommand*\listoflistings{\listof{codelisting}{List of Listings}}
\makeatother
\makeatletter
\makeatother
\makeatletter
\@ifpackageloaded{caption}{}{\usepackage{caption}}
\@ifpackageloaded{subcaption}{}{\usepackage{subcaption}}
\makeatother

\usepackage{bookmark}

\IfFileExists{xurl.sty}{\usepackage{xurl}}{} % add URL line breaks if available
\urlstyle{same} % disable monospaced font for URLs
\hypersetup{
  pdftitle={225 Building Environment Report},
  pdfauthor={NEU Seattle Devs (Hot Sauce)},
  colorlinks=true,
  linkcolor={blue},
  filecolor={Maroon},
  citecolor={Blue},
  urlcolor={Blue},
  pdfcreator={LaTeX via pandoc}}


\title{225 Building Environment Report}
\author{NEU Seattle Devs (Hot Sauce)}
\date{2025-03-27}

\begin{document}
\maketitle

\renewcommand*\contentsname{Table of contents}
{
\hypersetup{linkcolor=}
\setcounter{tocdepth}{3}
\tableofcontents
}

\begin{verbatim}


Sensor Data Summary
Start Time: 0:00 | End Time: 7:59
                       Temperature     Humidity      CO2      PM2.5    
                               min max      min max  min  max   min max
Sensor Name                                                            
Sensor 3 - Event Space          23  24       44  45  411  448     0   0
Sensor 6 - Room 216             24  24       43  43  482  572    10  10
Sensor 7 - Staff Space          21  24       45  50  472  586     0   0
\end{verbatim}

\section{Sensor Charts}\label{sensor-charts}

\subsection{Temperature}\label{temperature}

\begin{figure}[H]

{\centering \includegraphics[width=0.85\linewidth,height=\textheight,keepaspectratio]{./charts/temperature_chart.png}

}

\caption{Temperature}

\end{figure}%

\subsection{Humidity}\label{humidity}

\begin{figure}[H]

{\centering \includegraphics[width=0.85\linewidth,height=\textheight,keepaspectratio]{./charts/humidity_chart.png}

}

\caption{Humidity}

\end{figure}%

\subsection{CO2}\label{co2}

\begin{figure}[H]

{\centering \includegraphics[width=0.85\linewidth,height=\textheight,keepaspectratio]{./charts/co2_chart.png}

}

\caption{CO2}

\end{figure}%

\subsection{PM2.5}\label{pm2.5}

\begin{figure}[H]

{\centering \includegraphics[width=0.85\linewidth,height=\textheight,keepaspectratio]{./charts/pm_chart.png}

}

\caption{PM2.5}

\end{figure}%

\section{Comfort Level \& Indoor Climate
Score}\label{comfort-level-indoor-climate-score}

\begin{verbatim}
Indoor Comfort Score:  0.17
\end{verbatim}

\section{\texorpdfstring{\textbf{Indoor Comfort Score (ICS)
Calculation}}{Indoor Comfort Score (ICS) Calculation}}\label{indoor-comfort-score-ics-calculation}

\subsection{\texorpdfstring{\textbf{Comfort
Levels}}{Comfort Levels}}\label{comfort-levels}

\begin{itemize}
\tightlist
\item
  \textbf{Excellent (90-100) ✅} -- Ideal indoor conditions, highly
  comfortable.\\
\item
  \textbf{Good (75-89) 🙂} -- Slight deviations, but still
  comfortable.\\
\item
  \textbf{Moderate (50-74) 😐} -- Noticeable discomfort, but
  tolerable.\\
\item
  \textbf{Poor (25-49) 😕} -- Significant discomfort, action needed.\\
\item
  \textbf{Unacceptable (0-24) ❌} -- Severe discomfort, unhealthy
  conditions.
\end{itemize}

\begin{center}\rule{0.5\linewidth}{0.5pt}\end{center}

\subsection{\texorpdfstring{\textbf{Scoring
Logic}}{Scoring Logic}}\label{scoring-logic}

The \textbf{Indoor Comfort Score (ICS)} is based on four key indoor
environmental factors:

\begin{longtable}[]{@{}ll@{}}
\toprule\noalign{}
\textbf{Factor} & \textbf{Optimal Range} \\
\midrule\noalign{}
\endhead
\bottomrule\noalign{}
\endlastfoot
\textbf{Temperature (°F)} & \textbf{69.8 - 77} \\
\textbf{Humidity (\%)} & \textbf{40 - 60} \\
\textbf{CO2 (ppm)} & \textbf{400 - 800} \\
\textbf{PM2.5 (µg/m³)} & \textbf{0 - 12} \\
\end{longtable}

\subsubsection{\texorpdfstring{\textbf{How the Score is
Calculated}}{How the Score is Calculated}}\label{how-the-score-is-calculated}

1️⃣ \textbf{Ideal Conditions:}\\
- If a value falls \textbf{within the optimal range}, no penalty is
applied.

2️⃣ \textbf{Penalty for Deviations:}\\
- If a value is \textbf{outside the optimal range}, a
\textbf{sigmoid-based penalty} is applied:\\
- \textbf{Small deviations → minimal penalty}\\
- \textbf{Larger deviations → exponentially stronger penalty}

3️⃣ \textbf{Final Score Calculation:}\\
- The \textbf{Indoor Comfort Score (ICS)} is computed using a
\textbf{weighted geometric mean}:\\
- Ensures \textbf{no single factor dominates}\\
- Balances \textbf{all environmental parameters proportionally}\\
- Generates a \textbf{realistic comfort score}

\begin{center}\rule{0.5\linewidth}{0.5pt}\end{center}

\textbf{Sensor Models}: ESP8266, PMS5003(PM2.5), SHT31-D(Temp/Hum),
S8(CO2)\\
\textbf{Calibration Date}: January 15, 2025\\
\textbf{Sampling Interval}: 5 minutes




\end{document}
